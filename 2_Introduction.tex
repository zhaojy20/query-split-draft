\textcolor{blue}{
    Cardinality estimation (CE) is essential for cost-based query optimization because cardinality is used in the cost model. The error in cardinality estimation may lead the cost-based optimizer to sub-optimal plans.
}\par
\textcolor{blue}{
    Histogram-based cardinality estimation can make mistakes on complex queries due to the correlations in the predicates and the increasing estimation error with join size \cite{paper31}. However, such cardinality estimation method remains dominant in practice because of its low overhead. Many ideas have been raised to address the issue of cardinality estimation error, for example, samples \cite{paper19, paper22, paper23}, sketches \cite{paper55, PessimisticCE}, graphical models \cite{paper5}, and even machine learning \cite{MLCE}. However, they still cannot replace histogram-based ones because of their high online overhead or great efforts for pre-processing \cite{SampleReopt}.
}\par
    As it is challenging to solve the cardinality estimation problem \cite{MichaelStonebraker}, re-optimization \cite{Reopt, Pop, MichaelStonebraker, SampleReopt, QIE} is proposed to avoid the need for accurate cardinality estimation. Re-optimization first generates an initial execution plan at the beginning and then detects during run-time whether the actual behavior of a query plan becomes significantly different from what was expected. Then, re-optimization attempts to correct the execution plan whenever a significant deviation is found. In most implementations, re-optimization materializes a sub-tree of the global plan tree (sub-plan) and uses the statistics collected after materialization to refine the remaining part.\par
\textcolor{blue}{
    However, a problem with the current re-optimization techniques is that they rely heavily on the global plan. More specifically, re-optimization always chooses a sub-tree of the global plan to materialize. We call this way of re-optimization the sub-plan perspective. The problem is, due to the cardinality estimation error, the global plan is always far away from the optimal plan. Therefore, the terrible global plan influences the quality of the sub-plan. At its worst, the first few execution steps of the global plan may be bad decisions, and they cannot be corrected by re-optimization. For example, suppose the first-executed sub-plan deteriorates the execution time of the remaining query. It is impossible to correct that case because re-optimization will be triggered after this sub-optimal plan is executed.
}\par
\textcolor{blue}{
    To tackle the problem of current re-optimization, we propose a new re-optimization framework called \textit{query split}. Unlike traditional re-optimizations, \textit{query split} is designed in a sub-query perspective. Specifically, for select-projection-join (SPJ) queries, we first split them into several sub-queries before query optimization, then optimize and execute them sequentially to get the result of the original query. After materializing the sub-query results, we collect run-time statistics and use them in the query optimization for other sub-queries. From the sub-query perspective, we avoid the misleading global plan and decrease the problem scale of query optimization. As shown later in this paper, we get better execution plans than the sub-plan perspective by a proper sub-query splitting strategy.
}\par
\textcolor{blue}{
    Besides, the sub-query perspective provides a more robust overhead of materialization. In traditional query re-optimization, the times of materialization are unknown, depending on the current plan. For example, in mid-query re-optimization \cite{Reopt}, materialization happens at those blocked operators (e.g., sort and hash). In \textit{Pop} \cite{Pop}, materialization can also occur at the outer sides of the nest-loop join. This reactive pattern leads to too much or too little materialization overhead. In contrast, \textit{query split} decides where to materialize as soon as the query comes, decreasing the potential extreme overhead of materialization.
}\par
    Our experiment shows that \textit{query split} with the sub-query perspective beats sketch-based cardinality estimation approaches and traditional re-optimization on Join Order Benchmark \cite{JOB}. Moreover, \textit{query split} with the sub-query perspective can get an end-to-end latency which is very close to optimal.\par
\textcolor{blue}{
    The contribution of this paper can be summarized as follows:
    \begin{enumerate}[leftmargin = 15pt]
        \item We propose a new perspective for re-optimization called the sub-query perspective, which is opposite to the traditional sub-plan perspective. Sub-query perspective leads to a better plan for each sub-query and avoids the potential high overhead of materialization.
        \item We propose a re-optimization framework called \textit{query split} to cooperate with our sub-query perspective.
        \item Experimental results showed that \textit{query split} significantly outperformed existing approaches.
    \end{enumerate}
}\par
The rest of the paper is organized as follows. Section \ref{S2} motivates the sub-query perspective and \textit{query split}. Section \ref{S3} formally describes \textit{query split} and proves its correctness. Section \ref{S4} shows preliminary implementations of \textit{query split} with the sub-query perspective. Section \ref{S5} evaluates the performance of the \textit{query split}. Section \ref{S6} discusses how to extend \textit{query split} to general query forms. Section \ref{S7} gives a case study on \textit{query split} for a deep insight. Finally, Section \ref{S8} discusses related works, and Section \ref{S9} concludes this paper.