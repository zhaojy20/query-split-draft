\section{Proof of Theorem 1}
    This appendix shows the proof of Theorem 1.
    \begin{Theorem}
        Let q(\textbf{\textit{R}}, \textbf{\textit{S}}, \textbf{\textit{P}}) be an SP$J$ query, \textbf{\textit{Q}} be a set of subqueries of q. If \textbf{\textit{Q}}$\rightharpoonup_c$q, then the output of the replacement reconstruction algorithm is equal to the result of q.
    \end{Theorem}
    \begin{Proof}
        \ \newline 
        \indent To begin with, let us introduce several notations:
        \begin{enumerate}[leftmargin = 15pt]
            \item For a set of subqueries $\textbf{\textit{Q}}=\{q_1(\textbf{\textit{R}}_1,\textbf{\textit{S}}_1),...,q_n(\textbf{\textit{R}}_n,\textbf{\textit{S}}_n)\}$, we denote $R(\textbf{\textit{Q}})=\cup_{i=1}^n \textbf{\textit{R}}_i$, $S(\textbf{\textit{Q}})=\cup_{i=1}^n \textbf{\textit{S}}_i$.
            \item For a set of relations $\textbf{\textit{R}}=\{r_1,...,r_n\}$, we denote $\mathsf{X}_{r \in \textbf{\textit{R}}}=r_1 \times ... \times r_n$.
            \item For a SP$J$ query $q(\textbf{\textit{R}},\textbf{\textit{S}},\textbf{\textit{P}})$ and a set of subqueries $\textbf{\textit{Q}}$ of $q$, we denote the result of $q$ as $E(q)=\Pi_{\textbf{\textit{P}}}(\sigma_{\textbf{\textit{S}}}(\mathsf{X}_{r \in \textbf{\textit{R}}}))$, and the output of the replacement reconstruction algorithm as $E(\textbf{\textit{Q}}, \textbf{\textit{P}})$.
        \end{enumerate}\par
        \indent Notice that in both $E(q)$ and $E(\textbf{\textit{Q}}, \textbf{\textit{P}})$, the projection is performed at last over the same projection attribute set $\textbf{\textit{P}}$. Thus if the results before projection are same, the final results are same as well. Therefore, we can simplify notations of (3): For a SP$J$ query $q(\textbf{\textit{R}},\textbf{\textit{S}})$ and the set of subqueries $\textbf{\textit{Q}}$ of $q$, we denote the result of $q$ as $E(q)=\sigma_{\textbf{\textit{S}}}(\mathsf{X}_{r \in \textbf{\textit{R}}})$, and the output of the replacement reconstruction algorithm as $E(\textbf{\textit{Q}})$.
        \indent Under these notations, we can rewrite the theorem as: Given a SP$J$ query $q(\textbf{\textit{R}},\textbf{\textit{S}})$, and a set of subqueries $\textbf{\textit{Q}}$ of $q$, such that $\textbf{\textit{Q}} \rightharpoonup_c q$. Then we have $E(q)=E(\textbf{\textit{Q}})$.\newline
        \indent Without loss of generality, we assume that the names of all attributes in $\textbf{\textit{R}}$ are unique. Under such assumption, we do not need to consider the rename operation when modifying subqueries, and for simplicity we assume that the rename step is skipped.\newline
        \indent Now, we start to prove the rewritten theorem by induction on $|\textbf{\textit{Q}}|$.\newline
        \indent First, We prove the statement holds when $|\textbf{\textit{Q}}|=1$, in which case $\textbf{\textit{Q}}=\{q_1(\textbf{\textit{R}}_1,\textbf{\textit{S}}_1)\}$. Apparently the only way that $\textbf{\textit{Q}} \rightharpoonup_c q$ is $q_1=q$. So the statement clearly holds for $|\textbf{\textit{Q}}|=1$.\newline
        \indent Now, assume that the statement holds when $|\textbf{\textit{Q}}|=n-1$. We consider the case of $|\textbf{\textit{Q}}|=n$, $\textbf{\textit{Q}}=\{q_1(\textbf{\textit{R}}_1,\textbf{\textit{S}}_1),...,q_n(\textbf{\textit{R}}_n,\textbf{\textit{S}}_n)\}$.\newline
        \indent Without loss of generality, we denote the first executed subquery as $q_1(\textbf{\textit{R}}_1,\textbf{\textit{S}}_1)$ and discuss two cases: (1) $\forall i > 1, \textbf{\textit{R}}_1 \cap \textbf{\textit{R}}_i=\emptyset$ and (2) $\exists i > 1, s.t. \textbf{\textit{R}}_1 \cap \textbf{\textit{R}}_i \neq \emptyset$.\newline
        \textbf{Case 1}: We first execute $q_1$ and materialize its result as relation $m_1=E(q_1)$. Then, because $\forall i > 1, \textbf{\textit{R}}_1 \cap \textbf{\textit{R}}_i=\emptyset$, according to the algorithm, we have to add $m_1$ to the subquery result set $\textbf{\textit{L}}$ in \textbf{Loop(modify)} phase. After that, we remove $q_1$ from $\textbf{\textit{Q}}$ and have a new subquery set $\textbf{\textit{Q}}'=\{q_2(\textbf{\textit{R}}_2,\textbf{\textit{S}}_2),...,q_n(\textbf{\textit{R}}_n,\textbf{\textit{S}}_n)\}$ for the next iteration.\newline
        \indent We construct a SP$J$ query $q'(\textbf{\textit{R}}',\textbf{\textit{S}}')$, where $\textbf{\textit{R}}'= \cup_{i=2}^{n} \textbf{\textit{R}}_i$ and $\textbf{\textit{S}}'=\cup_{i=2}^{n} \textbf{\textit{S}}_i$. Apparently, $\textbf{\textit{Q}}' \rightharpoonup_c q'$ and as $|\textbf{\textit{Q}}'|=n-1$, by induction hypothesis, $E(q')=E(\textbf{\textit{Q}}')$.\newline
        \indent  According to \textbf{Merge} phase, final result is the Cartesian product on the elements in $\textbf{\textit{L}}$, so we have:
        $$E(\textbf{\textit{Q}})=\mathsf{X}_{r \in \textbf{\textit{L}}}=m_1 \times \mathsf{X}_{r \in (\textbf{\textit{L}} \setminus \{m_1\})}=E(q_1) \times E(\textbf{\textit{Q}}')=E(q_1) \times E(q')$$
        $$=\sigma_{\textbf{\textit{S}}_1}(\mathsf{X}_{r \in \textbf{\textit{R}}_1}) \times \sigma_{\textbf{\textit{S}}'}(\mathsf{X}_{r \in \textbf{\textit{R}}'})=\sigma_{\textbf{\textit{S}}_1 \cup \textbf{\textit{S}}'}(\mathsf{X}_{r \in \textbf{\textit{R}}})=\sigma_{\textbf{\textit{S}}}(\mathsf{X}_{r \in \textbf{\textit{R}}})=E(q)$$
        \textbf{Case 2}: We denote the set of subqueries that need to be modified after executing $q_1$ as $\textbf{\textit{W}}=\{q_k(\textbf{\textit{R}}_k,\textbf{\textit{S}}_k) \in \textbf{\textit{Q}}:k > 1,\textbf{\textit{R}}_1 \cap \textbf{\textit{R}}_k \neq \emptyset\}$.\newline
        \indent After we execute $q_1$ and materialize its result as relation $m_1$, we modify each $q_i \in \textbf{\textit{W}}$ and keep $\textbf{\textit{L}}=\emptyset$ in \textbf{Loop(modify)} phase. We denote these new-formed subqueries as $q'_i(\textbf{\textit{R}}'_i, \textbf{\textit{S}}'_i)$ and $\textbf{\textit{R}}'_i=\textbf{\textit{R}}_i \setminus \textbf{\textit{R}}_1 \cup \{m_1\}$, $\textbf{\textit{S}}'_i=\textbf{\textit{S}}_i$. These new-formed subqueries form a new set $\textbf{\textit{W}}'$.\newline
        \indent Now, $\textbf{\textit{Q}}$ becomes a new subquery set $\textbf{\textit{Q}}'=\textbf{\textit{Q}} \cup \textbf{\textit{W}}' \setminus \{q_1\} \setminus \textbf{\textit{W}}$. Because $\textbf{\textit{L}} = \emptyset$ at this point, when reconstruction finishes, we have $E(\textbf{\textit{Q}})=E(\textbf{\textit{Q}}')$.\newline
        \indent We construct a new query $q'(\textbf{\textit{R}}',\textbf{\textit{S}}')$, where $\textbf{\textit{R}}'=\textbf{\textit{R}} \setminus \textbf{\textit{R}}_1 \cup \{m_1\}$ and $\textbf{\textit{S}}'=\cup_{i=2}^{n} \textbf{\textit{S}}_i$. We will prove that $E(q')=E(\textbf{\textit{Q}}')$ and $E(q')=E(q)$, hence finishes the proof. To prove $E(q')=E(\textbf{\textit{Q}}')$, by induction hypothesis, we only need to show that $\textbf{\textit{Q}}' \rightharpoonup_c q'$ and $|\textbf{\textit{Q}}'|=n-1$.
        \begin{itemize}[leftmargin = 15pt]
            \item Apparently, $R(\textbf{\textit{Q}}')=R(\textbf{\textit{Q}}) \setminus \textbf{\textit{R}}_1 \cup \{m_1\}=\textbf{\textit{R}} \setminus \textbf{\textit{R}}_1 \cup \{m_1\}=\textbf{\textit{R}}'$. And because $S(\textbf{\textit{Q}}')=\cup_{i=2}^{n} \textbf{\textit{S}}_i=\textbf{\textit{S}}'$, $S(\textbf{\textit{Q}}')$ logical implies $\textbf{\textit{S}}'$, so $\textbf{\textit{Q}}' \rightharpoonup_c q'$.
            \item Notice that $\textbf{\textit{W}} \subseteq \textbf{\textit{Q}}$, $\{q_1\} \subseteq \textbf{\textit{Q}}$, $\textbf{\textit{W}}' \cap \textbf{\textit{Q}}=\emptyset$ and $|\textbf{\textit{W}}|=|\textbf{\textit{W}}'|$, so $|\textbf{\textit{Q}}'|=n-1$.
        \end{itemize}\par
        \indent By using the induction hypothesis, we get $E(q')=E(\textbf{\textit{Q}}')$.\newline
        \indent At last, we need to prove $E(q')=E(q)$:
        $$E(q')=\sigma_{\textbf{\textit{S}}'}(\mathsf{X}_{r \in \textbf{\textit{R}}'})=\sigma_{\textbf{\textit{S}}'}(\mathsf{X}_{r \in (\textbf{\textit{R}}' \setminus \{m_1\})} \times m_1)=\sigma_{\textbf{\textit{S}}'}(\mathsf{X}_{r \in (\textbf{\textit{R}} \setminus \textbf{\textit{R}}_1)} \times m_1)$$
        \indent Since $m_1$ is the execution result of $q_1$, $m_1=\sigma_{\textbf{\textit{S}}_1}(\mathsf{X}_{r \in \textbf{\textit{R}}_1})$, so we have:
        $$E(q')=\sigma_{\textbf{\textit{S}}'}(\mathsf{X}_{r \in (\textbf{\textit{R}} \setminus \textbf{\textit{R}}_1)} \times \sigma_{\textbf{\textit{S}}_1}(\mathsf{X}_{r \in \textbf{\textit{R}}_1}))=\sigma_{\textbf{\textit{S}}}(\mathsf{X}_{r \in \textbf{\textit{R}}})=E(q)$$
        \indent Thus, $E(q)=E(\textbf{\textit{Q}})$ and the statement holds for $|\textbf{\textit{Q}}|=n$.
    \end{Proof}